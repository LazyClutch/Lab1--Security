\section*{Task 3: Authentication}
The third and last task of the lab was about authentication, and the exercises were of a theoretical rather than practical nature.
\subsection*{Exercises:}
\begin{enumerate}
\item % (a)
Hashing a list of passwords with a one-way hash function, say $f$, rather than having the passwords stored in cleartext is a method for reducing the impact of a database leak. Authentication given a password $p$ in such a system is done by computing the hash $f(p)$ of the password and comparing it with the hash for the specific user stored in the database.

In the event that the password hashes are exposed to an attacker, there is no trivial way for him or her to retrieve the passwords corresponding to the compromised password hashes, by the definition of a one-way hash function. Thus, there is no trivial way of authenticating despite having access to both the hash function and the hashes. If an attacker finds the passwords stored in cleartext, on the other hand, he or she can authenticate as any of the affected users on the system. Furthermore, because passwords are frequently reused in more than one system, the attacker might be able to authenticate falsely on other systems as well. In other words, the impact of a cleartext password leak could encompass more than just the attacked system.

While there is no \emph{trivial} way of finding a password corresponding to a specific hash value, there is a cumbersome but in practice feasible way of doing so: password guessing. The idea is to repeatedly come up with a guess for the password, hash it, and compare the result with the hash value in question, continuing until a password guess with the specific hash is found. The hash values for guesses for common passwords can be precomputed into a so-called rainbow table that allow for these passwords to be looked up given their hash value.

Salting is a method for making password guessing more difficult. In addition to storing a hash value for each user, a random string $s$---the salt---is stored for each user as well. Authentication given a password $p$ is done by computing the hash of a combination of the password and the salt, such as $s + p$ where ``$+$'' denotes a concatenation. The resulting hash, $f(s + p)$, is then compared with the hash value in the database.

If an attacker gets hold of the database, which contains a salt and hash for each user, and aims for finding the password of \emph{any} of the users, the addition of the salt will make password guessing more difficult. Let $n$ be the number of users. Under the assumption that each user has a unique salt, the attacker will have to compute $n$ hashes for each password guess, compared to the single computation that is required with hashing alone. Salting offers increased security also against attacks with precomputed tables, seeing as one table would be needed for each possible salt in order to cover the same passwords as without salting.
\item % (b)
Assuming that the salt and salted hash are known to an attacker, salting may make the attack on a specific account more difficult, depending on whether precomputed tables are used.

If precomputed tables are used and the attacker did not choose the specific user because the attacker had a precomputed table for that specific salt, salting will make the attack more difficult in that more precomputed tables will be required.

Without precomputation, however, each password guess will require one computation of a hash value, assuming that the hash is computed like mentioned before: $f(s + g)$ for hash function $f$, salt $s$, password guess $g$, and concatenation operator $+$. The only difference in the case that salting is not used is that no salt has to be concatenated; the hash function still has to be evaluated once for each guess. Under the assumption that concatenating the string takes negligible time compared to evaluating the hash function, salting does not make the attack in question harder.
\item \highergradesonly
  \begin{itemize}
    \item
    \item
  \end{itemize}
\item % (d)
  \begin{itemize}
    \item
      While the fingerprint sensor of the iPhone 5s can evidently be deceived in a variety of ways, there are at least two reasons that it is a good addition to the iPhone security. The first reason is that the use of the fingerprint sensor in conjunction with more traditional authentication methods such as PIN codes adds another layer for the attacker to get past. The other reason, while not as convincing as the first, is that the sensor might provide a strong enough protection for some users, by the principle of adequate protection.
    \item
      The principle of authenticating with something you have could be leveraged to increase the security of the phone unlocking mechanism drastically. One could imagine that the presence of a physical device---a \textit{token}---capable of communicating with the phone in a wireless manner would be required to unlock the phone. With a challenge-response system, the token could prove its presence without opening up the possibility of a replay attack.

In this, the time of smart devices, the token need not be a piece of special-purpose hardware with the only purpose of authenticating the possessor to the phone; for instance, a smartwatch could be assigned the task of authentication on the phone.

While requiring a separate device of some sort to unlock the phone is obviously beneficial in the case that the phone but not the device is lost, a downside of the system is apparent in the opposite scenario: losing the device but not the phone. In that case, the phone is rendered unusable. There being ramifications of an authentication system is not unique to those based on the principle of what you have though; the situation of losing one's device can be likened to that of forgetting  one's PIN code, which also renders the phone unusable.
  \end{itemize}
\end{enumerate}
